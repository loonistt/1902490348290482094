\section*{Приложение 1} \label{application1}
\addcontentsline{toc}{section}{Приложение 1}
\pagestyle{empty}

\setcounter{figure}{0}

Полный программный код реализации волнового алгоритма:

\begin{lstlisting}
# Полный код программы (main.py)
import pygame
import sys
import random

# Инициализация PyGame
pygame.init()
WIDTH, HEIGHT = 800, 600
screen = pygame.display.set_mode((WIDTH, HEIGHT))
clock = pygame.time.Clock()

# Цвета
WHITE = (255, 255, 255)
BLACK = (0, 0, 0)
BLUE = (0, 120, 255)
GREEN = (52, 199, 89)
RED = (255, 59, 48)
GRAY = (200, 200, 200)
YELLOW = (255, 200, 0)
PURPLE = (175, 82, 222)

# Класс Maze: генерация лабиринтов
class Maze:
    def __init__(self, width=21, height=15):
        self.width = width
        self.height = height
        self.start = (1, 1)
        self.end = (width - 2, height - 2)
        self.grid = self.generate()
    
    def generate(self):
        # ... (полный код класса Maze из документа)
    
    def update_grid_positions(self):
        # ... (полный код метода)

# Класс WaveSolver: реализация волнового алгоритма
class WaveSolver:
    def __init__(self, maze):
        # ... (полный код класса WaveSolver из документа)
    
    def solve(self):
        # ... (полный код метода)
    
    def reconstruct_path(self):
        # ... (полный код метода)

# Класс App: интерфейс и визуализация
class App:
    def __init__(self):
        # ... (полный код класса App из документа)
    
    def handle_events(self):
        # ... (полный код метода)
    
    def draw(self):
        # ... (полный код метода)
    
    def update(self):
        # ... (полный код метода)
    
    def run(self):
        # ... (полный код метода)

# Главная функция
if __name__ == "__main__":
    app = App()
    app.run()
\end{lstlisting}